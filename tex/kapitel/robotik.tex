\chapter{Robotik} % (fold)
\label{cha:robotik}

Lego Roboter

\section{Linienfolger} % (fold)
\label{sec:linienfolger}

Die Idee des Linienfolgers stammt von der Rescue-League des Robocups, einem Wettbewerb in dem es ursprünglich um einen Roboterfußball-Wettkampf ging. Bei der Rescue League müssen die Roboter einer Linie folgen und am Ende ein „Opfer“ erkennen und bergen. Erschwert wird dies durch Rampen, Lücken und Hindernisse auf und neben der Linie.

In vereinfachter Form geht es zunächst nur um die Linie und so muss der Robotor einfach einer schwarzen Linie folgen.

% section linienfolger (end)

\section{Lego Mindstorm EV3} % (fold)
\label{sec:lego_mindstorm_ev3}

leJOS ist die nunmehr dritte Generation des Java-Betriebssystems für den programmierbaren Lego-Mindstorms Stein EV3. Diese Software erlaubt die Steuerung von diversen Lego-Konstruktionen in Java. Unter anderem können so verschiedene Sensoren ausgelesen und Motoren angesteuert werden.

% section lego_mindstorm_ev3 (end)

50:
00: G01: L10: G11: L
100:00: R01: L10: R11: L
150:00: L01: L10: G11: L
200:00: G01: L10: R11: L
250:00: G01: L10: R11: R
300:00: L01: L10: G11: R
350:00: G01: L10: L11: R
400:00: G01: L10: L11: R
450:00: G01: L10: L11: R
500:00: G01: G10: R11: R
550:00: G01: L10: R11: R
% chapter robotik (end)