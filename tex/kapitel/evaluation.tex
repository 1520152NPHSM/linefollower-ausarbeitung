\chapter{Evaluation} % (fold)
\label{cha:evaluation}

Für die Erprobung des Q-Learning-Algorithmuses wurden zwei Linienfolger und eine Teststrecke konzipiert und zusammengebaut. Der Kurs (siehe Abbildung \ref{fig:kurs}) besteht aus verschiedenen Geraden- und Kurvenstücken von variablem Schwierigkeitsgrad.

\begin{figure}[!htb]
	\centering
	\includegraphics[width=8cm]{kurs.png}
	\caption{Kurs}
	\label{fig:kurs}
\end{figure}

Die beiden Linienfolger unterscheiden sich in der Anzahl und der Anordnung der Farbsensoren. Der erste Linienfolger $L_0$ verfügt über drei Farbsensoren, die nebeneinander angeordnet sind. Die schwarze Linie besitzt dieselbe Breite wie der Lichtkegel eines Farbsensors. Sommit ist der Linienfolger optimal Positioniert falls nur der mittlere Sensor Schwarz und die beiden außenliegenden Sensoren Weiß erkennen. Die Tabelle \ref{tab:zustaende_drei_sensoren} zeigt alle eintretbaren Zustände.

\begin{table}[h]
  \caption{Zustände bei drei Sensoren}
  \label{tab:zustaende_drei_sensoren}
  \renewcommand{\arraystretch}{1.2}
  \centering
  \sffamily
  \begin{footnotesize}
    \begin{tabular}{l l l}
    \toprule
    \textbf{Zustand} & \textbf{Beschreibung} & \textbf{Belohnung}\\
    \midrule
    $s_0$	&	Alle Sensoren erkennen Weiß.	&	0\\ % 000
    $s_1$	&	Der rechte Sensor erkennt Schwarz.	&	0\\ % 001
    $s_2$	&	Der mittlere Sensor erkennt Schwarz.	&	0\\ % 010
    $s_3$	&	Der mittlere und rechte Sensor erkennt Schwarz.	&	0\\ % 011
    $s_4$	&	Der linke Sensor erkennt Schwarz.	&	0\\ % 100
    $s_5$	&	Der linke und rechte Sensor erkennt Schwarz.	&	0\\ % 101
    $s_6$	&	Der linke und mittlere Sensor erkennt Schwarz.	&	0\\ % 110
    $s_7$	&	Alle Sensoren erkennen Schwarz.	&	0\\ % 111
    \bottomrule
    \end{tabular}
  \end{footnotesize}
  \rmfamily
\end{table}

Im Gegensatz zum ersten Linienfolger besitzt der zweite Linienfolger $L_1$ zwei Farbsensoren, die angewinkelt angeordnet sind. Sommit besitzt er andere Zustände als $L_0$, diese sind in der Tabelle \ref{tab:zustaende_zwei_sensoren} beschrieben.

\begin{table}[h]
  \caption{Zustände bei zwei Sensoren}
  \label{tab:zustaende_zwei_sensoren}
  \renewcommand{\arraystretch}{1.2}
  \centering
  \sffamily
  \begin{footnotesize}
    \begin{tabular}{l l l}
    \toprule
    \textbf{Zustand} & \textbf{Beschreibung} & \textbf{Belohnung}\\
    \midrule
    $s_0$	&	Beide Sensoren erkennen Weiß.	&	0\\
    $s_1$	&	Der rechte Sensor erkennt Schwarz.	&	0\\
    $s_2$	&	Der linke Sensor erkennt Schwarz.	&	0\\
    $s_3$	&	Beide Sensoren erkennen Schwarz.	&	1\\
    \bottomrule
    \end{tabular}
  \end{footnotesize}
  \rmfamily
\end{table}

Zur Erreichung eines anderen Zustandes können beide Linienfolger folgende Aktionen (siehe Tabelle \ref{tab:aktionen}) ausführen. Die Art der Ausführung unterscheidet sich nicht bei $L_0$ oder $L_1$.

\begin{table}[h]
  \caption{Aktionen}
  \label{tab:aktionen}
  \renewcommand{\arraystretch}{1.2}
  \centering
  \sffamily
  \begin{footnotesize}
    \begin{tabular}{l l}
    \toprule
    \textbf{Aktion} & \textbf{Beschreibung}\\
    \midrule
    $a_0$	&	Linkskurve\\
    $a_1$	&	Ge­ra­de­aus­fahrt\\
    $a_2$	&	Rechtskurve\\
    \bottomrule
    \end{tabular}
  \end{footnotesize}
  \rmfamily
\end{table}

Beide Linienfolger nutzen für das Q-Learning die gleichen Parameter, um die Vergleichbarkeit der Ergebnisse zu gewährleisten.

\begin{table}[h]
  \caption{Q-Learning-Parameter}
  \label{tab:q-parameter}
  \renewcommand{\arraystretch}{1.2}
  \centering
  \sffamily
  \begin{footnotesize}
    \begin{tabular}{l l}
    \toprule
    \textbf{Parameter} & \textbf{Wert}\\
    \midrule
    $\alpha$	&	$0.9$\\
    $\gamma$	&	$0.2$\\
    $\epsilon$	&	$0.01$\\
    \bottomrule
    \end{tabular}
  \end{footnotesize}
  \rmfamily
\end{table}

Die optimale Strategie für $L_0$ ist im Graph \ref{fig:zwei_sensoren_optimale_strategie} angegeben, die für $L_1$ in Graph \ref{fig:drei_sensoren_optimale_strategie}.

\begin{figure}
\centering
\begin{tikzpicture}[->,>=stealth',shorten >=1pt,auto,node distance=3cm,
                    thick,main node/.style={circle,draw,font=\sffamily\Large\bfseries}]

  \node[main node] (1) {$s_1$};
  \node[main node] (2) [below left of=1] {$s_0$};
  \node[main node] (3) [below right of=2] {$s_3$};
  \node[main node] (4) [below right of=1] {$s_2$};

  \path[every node/.style={font=\sffamily\small}]
    (1) edge node [left] {$a_0$} (3)
    (2) edge node [left] {$a_0$, $a_1$, $a_2$} (3)
    (3) edge [loop below] node {$a_1$} (3)
    (4) edge node [left] {$a_2$} (3);
\end{tikzpicture}
\caption{Optimale Strategie bei zwei Sensoren} 
\label{fig:zwei_sensoren_optimale_strategie}
\end{figure}

\begin{figure}
\centering
\begin{tikzpicture}[->,>=stealth',shorten >=1pt,auto,node distance=3cm,
                    thick,main node/.style={circle,draw,font=\sffamily\Large\bfseries}]

  \node[main node] (1) {$s_2$};
  \node[main node] (2) [below left of=1] {$s_1$};
  \node[main node] (3) [below right of=1] {$s_3$};
  \node[main node] (4) [below left of=2] {$s_0$};
  \node[main node] (5) [below right of=3] {$s_4$};
  \node[main node] (6) [below right of=4] {$s_7$};
  \node[main node] (7) [below left of=5] {$s_5$};
  \node[main node] (8) [below right of=6] {$s_6$};


  \path[every node/.style={font=\sffamily\small}]
    % (1) edge node [left] {$a_1$} (4)
    %     edge [bend right] node[left] {0.3} (2)
    %     edge [loop above] node {0.1} (1)
    % (2) edge node [right] {0.4} (1)
    %     edge node {0.3} (4)
    %     edge [loop left] node {0.4} (2)
    %     edge [bend right] node[left] {0.1} (3)
    % (3) edge node [right] {0.8} (2)
    %     edge [bend right] node[right] {0.2} (4)
    % (4) edge node [left] {0.2} (3)
    %     edge [loop right] node {0.6} (4)
    %     edge [bend right] node[right] {0.2} (1);
    (1) edge [loop above] node {$a_1$} (1)
    (2) edge node [above] {$a_2$} (3)
        edge [loop left] node {$a_2$} (2)
    (3) edge node [right] {$a_2$} (1)
    (4) edge node [left] {$a_2$} (2)
        edge [bend above] node [above] {$a_0$} (5)
    (5) edge [bend left] node {$a_0$} (8)
        edge [bend right] node [right] {$a_0$} (1)
    (6) edge node [left] {$a_0$} (8)
        edge [bend left] node [left] {$a_2$} (3)
    (7) edge node [left] {$a_0$} (5)
        edge [bend left] node [left] {$a_2$} (2)
    (8) edge [bend right] node [left below] {$a_0$} (1);
\end{tikzpicture}
\caption{Optimale Strategie bei drei Sensoren} 
\label{fig:drei_sensoren_optimale_strategie}
\end{figure}

Bei beiden Linienfolgern wurden die optimalen Aktionen für alle Zustände alle 50 Lernschritte protokoliert.

Tabelle für $L_0$ hier.

Tabelle für $L_1$ hier.

Die gelerneten Strategien weichen nach 2000 Trainingszyklen bei beiden Linienfolgern weitgehend von der optimalen Strategie ab. \par
Bei $L_0$ ist die gewinkelte Anordnung der Sensoren ein Problem, da die beiden Lichtsensoren nicht dasselbe Lichtkegelzentrum haben. Die beiden Lichtkegel überschneiden sich nur in den Außenbereichen. Dies läst sich aufgrund der vorgegebenen Legobauteile nicht vermeiden. Dies führt zu einer fehlerhaften Erkennung von $s_3$. Des Weiteren können die Sensoren aufgrund der Bauweise nicht star befestigt werden und schwanken sommit im Berreich der Lichtkegel. Dies führt zu Messungenauigkeiten. \par
Der Einsatz von drei Sensoren in $L_1$ führt zu neuen Schwierigkeiten, da für manche Zustände die optimale Lösung von der auftretenden Situation abhänig ist und sommit eine einmal gelerente Lösung zu einem späteren Zeitpunkt zu einer faschen Lösung führen kann. Auch führt der größere Zustandsraum zu ... \par
Anzumerken ist das die Zustände nach über 2000 Lernzyklen noch übermässig fluktuieren und sich nicht einpendeln.

- Probleme -> Ursachen -> Lösung (falls vorhanden)  

% chapter evaluation (end)