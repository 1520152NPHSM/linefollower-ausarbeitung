\chapter{Einleitung} % (fold)
\label{cha:einleitung}

Roboter auf herkömmliche Weise zu Programmieren kann ein langwieriger Prozess sein, in dem für alle vorhandene Sensordaten immer die passende Aktion ausgeführt werden muss. Ein anderer Ansatz für dieses Problem ist der Einsatz von Maschinellem Lernen um ein Problem darzustellen. In vielen Gebieten des sogenannten Robocups wird bereits damit gearbeitet und auch im realen Umfeld sieht man schon erste Anwendung bei z.B. autonom fahrenden Autos, oder der Bilderkennung. Das Interesse am Maschinellen Lernen steigt also stetig. Konkret geht es in dieser Arbeit um eine Form des Reinforcement Learning, das Q-Learning. Dabei lernt der Roboter selbst, auf Basis der eingelesenen Sensordaten, was für den aktuellen Status die am besten geeignetste Aktion ist. Somit ist es dem Programmierer lediglich überlassen die Basis zu schaffen.\par
Die genaue Aufgabe, mit der sich in dieser Arbeit beschäftigt wurde, war dieses Verfahren auf einen Linienfolger anzuwenden. Um aus dieser eigentlich einfachen Aufgabe mehr Nutzten zu ziehen, wurden gleich zwei Roboter gebaut, die auf eine unterschiedliche Anzahl von Sensoren zurückgreifen.\par
In der Projektarbeit werden die theoretischen Grundlagen sowie die Java-Im\-ple\-men\-tierung und der Aufhau der beiden Roboter beschrieben. Das Ergebnis der Arbeit sind 2 unterschiedliche Umsetzungen der Aufgabe des Linienfolgens, die verglichen und im Bezug auf das gegebene Szenario bewertet wurden.

% chapter einleitung (end)