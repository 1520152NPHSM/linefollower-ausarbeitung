\chapter{Robotik} % (fold)
\label{cha:robotik}

Die Robotik ist ein umfassendes Themengebiet mit einer Vielzahl an un\-ter\-schied\-lichen Disziplinen. Im Rahmen dieser Lehrveranstaltung wurde der Fokus auf das Verfolgen einer Linie gesetzt.

\section{Linienfolger} % (fold)
\label{sec:linienfolger}

Die Idee des Linienfolgers stammt von der Rescue-League des Robocups, einem Wettbewerb in dem es ursprünglich um Roboterfußball ging. Bei der Rescue League müssen die Roboter einer Linie folgen und am Ende ein \enquote{Opfer} erkennen und bergen. Erschwert wird dies durch Rampen, Lücken und Hindernisse auf und neben der Linie.\par
Im Rahmen dieser Ausarbeitung wird auf den Einsatz von Hindernissen verzichtet. Der Linienfolger muss nur der Linie folgen.

% section linienfolger (end)

\section{Lego Mindstorms EV3} % (fold)
\label{sec:lego_mindstorm_ev3}

leJOS ist die dritte Generation des Java-Betriebssystems für den programmierbaren Lego-Mindstorms-Stein EV3. Diese Software erlaubt die Steuerung von diversen Lego-Konstruktionen in Java. Unter anderem können so verschiedene Sensoren ausgelesen und Motoren angesteuert werden.\par
Da der Linienverlauf variabel ist und sich nicht alle Streckenkonstruktionen im Vorraus durch geeignete Programmierkonstrukte und -algorithmen abdecken lassen können, ist der Einsatz von Maschinellem Lernen in der Robotik von Interesse. Dies bietet den Vorteil, dass der Roboter durch Einsatz geeigneter Techniken von alleine eine geeignete Lösung finden kann.

% section lego_mindstorm_ev3 (end)

% chapter robotik (end)