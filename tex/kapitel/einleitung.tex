\chapter{Einleitung} % (fold)
\label{cha:einleitung}

Roboter auf herkömmliche Weise zu Programmieren kann ein langwieriger Prozess sein, in dem für alle vorhandene Sensordaten immer die passende Aktion ausgeführt werden muss. Ein anderer Ansatz für dieses Problem ist der Einsatz von Maschinellem Lernen um das Problem zu lösen. In vielen Gebieten des Robocups wird bereits damit gearbeitet und auch im realen Umfeld sieht man schon erste Anwendung bei z.\,B. autonom fahrenden Autos und der Bilderkennung. Das Interesse am Maschinellen Lernen steigt stetig. Konkret geht es in dieser Arbeit um eine Form des Reinforcement Learning, das Q-Learning. Dabei lernt der Roboter selbst, auf Basis der eingelesenen Sensordaten, was für den aktuellen Status die am besten geeignetste Aktion ist. Somit ist es dem Programmierer überlassen die Zustände der Welt zu beschreiben.\par
Die genaue Aufgabe, mit der sich diese Arbeit beschäftigt, ist dieses Verfahren auf einen Linienfolger anzuwenden. Zur Evaluation der Aufgabe wurden zwei Linienfolger unterschiedlicher Bauweise und Sensoranzahl konstruiert.\par
In dieser Ausarbeitung werden die theoretischen Grundlagen, sowie die Java-Im\-ple\-men\-tierung und der Aufbau der beiden Roboter beschrieben. Das Ergebnis der Arbeit sind zwei unterschiedliche Linienfolger, die verglichen und im Bezug auf das gegebene Szenario bewertet werden.

% chapter einleitung (end)